% -----------------------------------------------
% Template for SMC 2020
% adapted from previous SMC paper templates
% -----------------------------------------------

\documentclass[dvipsnames]{article}
\usepackage{smc2020}
\usepackage{times}
\usepackage{ifpdf}
\usepackage[english]{babel}
\usepackage{cite}
\usepackage{multirow}
%%%%%%%%%%%%%%%%%%%%%%%% Some useful packages %%%%%%%%%%%%%%%%%%%%%%%%%%%%%%%
%%%%%%%%%%%%%%%%%%%%%%%% See related documentation %%%%%%%%%%%%%%%%%%%%%%%%%%
\usepackage{amsmath} % popular packages from Am. Math. Soc. Please use the 
\usepackage{amssymb}
\usepackage{cases}
% related math environments (split, subequation, cases,
%\usepackage{amsfonts}% multline, etc.)
%\usepackage{bm}      % Bold Math package, defines the command \bf{}
%\usepackage{paralist}% extended list environments
%%subfig.sty is the modern replacement for subfigure.sty. However, subfig.sty 
%%requires and automatically loads caption.sty which overrides class handling 
%%of captions. To prevent this problem, preload caption.sty with caption=false 
%\usepackage[caption=false]{caption}
%\usepackage[font=footnotesize]{subfig}


%user defined variables
\def\papertitle{Real-time Implementation of a Physical Model of the Tromba Marina} %using Finite-Difference Schemes}
\def\firstauthor{Silvin Willemsen}
\def\secondauthor{Stefan Bilbao}
\def\thirdauthor{Michele Ducceschi}
\def\fourthauthor{Stefania Serafin}
% \def\firstauthor{Silvin Willemsen, Stefania Serafin}
% \def\secondauthor{Stefan Bilbao and Michele Ducceschi}

% adds the automatic
% Saves a lot of output space in PDF... after conversion with the distiller
% Delete if you cannot get PS fonts working on your system.

% pdf-tex settings: detect automatically if run by latex or pdflatex
\newif\ifpdf
\ifx\pdfoutput\relax
\else
   \ifcase\pdfoutput
      \pdffalse
   \else
      \pdftrue
\fi

\ifpdf % compiling with pdflatex
  \usepackage[pdftex,
    pdftitle={\papertitle},
    pdfauthor={\firstauthor, \secondauthor, \thirdauthor, \fourthauthor},
    bookmarksnumbered, % use section numbers with bookmarks
    pdfstartview=XYZ % start with zoom=100% instead of full screen; 
                     % especially useful if working with a big screen :-)
   ]{hyperref}
  %\pdfcompresslevel=9

  \usepackage[pdftex]{graphicx}
  % declare the path(s) where your graphic files are and their extensions so 
  %you won't have to specify these with every instance of \includegraphics
  \graphicspath{{./figures/}}
  \DeclareGraphicsExtensions{.pdf,.jpeg,.png}

  \usepackage[figure,table]{hypcap}

\else % compiling with latex
  \usepackage[dvips,
    bookmarksnumbered, % use section numbers with bookmarks
    pdfstartview=XYZ % start with zoom=100% instead of full screen
  ]{hyperref}  % hyperrefs are active in the pdf file after conversion

  \usepackage[dvips]{epsfig,graphicx}
  % declare the path(s) where your graphic files are and their extensions so 
  %you won't have to specify these with every instance of \includegraphics
  \graphicspath{{./figures/}}
  \DeclareGraphicsExtensions{.eps}

  \usepackage[figure,table]{hypcap}
\fi

%setup the hyperref package - make the links black without a surrounding frame
\hypersetup{
    colorlinks,%
    citecolor=black,%
    filecolor=black,%
    linkcolor=black,%
    urlcolor=black
}


% Title.
% ------
\title{\papertitle}

% Authors
% Please note that submissions are NOT anonymous, therefore 
% authors' names have to be VISIBLE in your manuscript. 
%
% Single address
% To use with only one author or several with the same address
% ---------------
%\oneauthor
%   {\firstauthor} {Affiliation1 \\ %
%     {\tt \href{mailto:author1@smcnetwork.org}{author1@smcnetwork.org}}}

%Two addresses
%--------------
\twoauthors
  {\firstauthor{ and }\fourthauthor} {Multisensory Experience Lab, CREATE \\ Aalborg University Copenhagen \\
    {\tt \href{mailto:sil@create.aau.dk}{\{sil, sts\}@create.aau.dk}}}
  {\secondauthor{ and }\thirdauthor} {Acoustics and Audio Group \\ University of Edinburgh \\ %
    {\tt {\{s.bilbao, michele.ducceschi\}@ed.ac.uk}}}

% Three addresses
% --------------
%  \threeauthors
%   {\firstauthor} {Affiliation1 \\ %
%      {\tt \href{mailto:author1@smcnetwork.org}{author1@smcnetwork.org}}}
%   {\secondauthor} {Affiliation2 \\ %
%      {\tt \href{mailto:author2@smcnetwork.org}{author2@smcnetwork.org}}}
%   {\thirdauthor} { Affiliation3 \\ %
%      {\tt \href{mailto:author3@smcnetwork.org}{author3@smcnetwork.org}}}
% \fourauthors
%   {\firstauthor} {Multisensory Experience Lab, CREATE \\ Aalborg University Copenhagen \\ %
%      {\tt \href{mailto:sil@create.aau.dk}{sil@create.aau.dk}}}
%   {\secondauthor} {Acoustics and Audio Group \\ University of Edinburgh \\ %
%      {\tt \href{mailto:s.bilbao@ed.ac.uk}{s.bilbao@ed.ac.uk}}}
%     {\thirdauthor} {Acoustics and Audio Group \\ University of Edinburgh \\ %
%      {\tt \href{mailto:michele.ducceschi@ed.ac.uk}{michele.ducceschi@ed.ac.uk}}}
%   {\fourthauthor} {Multisensory Experience Lab, CREATE \\ Aalborg University Copenhagen \\ %
%      {\tt \href{mailto:sts@create.aau.dk}{sts@create.aau.dk}}}

% ***************************************** the document starts here ***************
\usepackage{xcolor}
\def\SBcomment[#1]{\textcolor{Red}{#1}}
\def\SWcomment[#1]{\textcolor{Orange}{#1}}
\def\MDcomment[#1]{\textcolor{Blue}{#1}}
\def\SScomment[#1]{\textcolor{OliveGreen}{#1}}

\begin{document}
%
\capstartfalse
\maketitle
\capstarttrue
%
\begin{abstract}
You can place comments in colour if you want :) Like \SBcomment[this (Stefan)], \MDcomment[this (Michele)] or \SScomment[this (Stefania)].
\end{abstract}
%

\section{Introduction}\label{sec:introduction}
Physical modelling for sound synthesis knows a long history. Throughout the past few decades, several techniques have been developed to simulate real-world instruments, including mass-spring systems~\cite{cadoz79, cadoz83, cadoz1993cordis}, digital waveguides~\cite{smith1992physical} and modal synthesis~\cite{morrison1993mosaic}. 

Finite-difference time-domain (FDTD) methods have first been used for sound synthesis in~\cite{Ruiz1969, Hiller1971, Hiller2}, later by other authors in~\cite{Chaigne92, Chaigne} and elaborated upon in~\cite{bilbao2009numerical, Bilbao2018:Tutorial}. As compared to other techniques, FDTD methods are more computationally expensive, but easily generalisable and accurate. Our goal is to implement these accurate techniques in real-time and thereby make the simulations playable for the users. 

The behaviour of musical instruments can be well described by partial-differential equations (PDEs). \textbf{$\leftarrow$ probably used exactly like this before}

Simulating musical instruments using physical models has several purposes. One of these is the resurrection of instruments that, due to fragility, rarity or value, can not be played anymore. In this paper, we present a real-time implementation of a physical model of the tromba marina, a tall bowed monochord from medieval Europe~\cite{encyclopaedia2020}. One string rests on a loose bridge that rattles against the body when the string is bowed. Interestingly, this creates a sound that has trumpet-like qualities, hence the name. 

For the non-linear collisions present in the instrument, a method recently proposed in the field of audio by Ducceschi and Bilbao in~\cite{Ducceschi2019} has been used. As it replaces the use of iterative solvers, such as Newton-Raphson, it perfectly serves our goal of creating a real-time implementation of the tromba marina.

This paper is structured as follows: Section \ref{sec:models} presents the models used, Section \ref{sec:disc} shows 

\section{Models}\label{sec:models}
The tromba marina can be subdivided into three main components: the string, the bridge and the body. In this section, the PDEs of the different components in isolation will be given in the form
\begin{equation}\label{eq:PDEform}
    \mathcal{L}u = 0,
\end{equation}
with linear partial differential operator $\mathcal{L}$ and $u = u(\boldsymbol{x},t)$ describes the component state over time $t$ and space $\boldsymbol{x}\in\mathcal{D}$, where the dimensions of domain $\mathcal{D}$ depend on the component at hand. Furthermore, the following shorthand notation is used
\begin{equation}
     \frac{\partial}{\partial t} = \partial_t, \ \ \frac{\partial^2}{\partial t^2} = \partial^2_t, \ \ \frac{\partial^2}{\partial x^2} = \partial^2_x,\ \ \frac{\partial^3}{\partial t\partial x^2} = \partial_t\partial^2_x,
\end{equation}
for derivatives with respect to time $t$ and space $x$. As different models share variable names, subscripts `$\text{s}$', `$\text{m}$' and `$\text{p}$' are added to denote that the variable either applies to the string, bridge (mass) or body (plate).

\subsection{Bowed Stiff string}
Consider a damped stiff string of length $L$ (m), with domain $\mathcal{D} = \mathcal{D}_\text{s} = [0,L]$ and state variable $u = u_\text{s}(x,t)$. We define the operator $\mathcal{L} = \mathcal{L_\text{s}}$ as
\begin{equation}
    \mathcal{L}_\text{s} = \rho_\text{s} A \partial_t^2 - T\partial_x^2 + E_\text{s}I\partial_x^4+2\rho_\text{s} A\sigma_{0,\text{s}}\partial_t-2\rho_\text{s} A\sigma_{1,\text{s}}\partial_t\partial_x^2,
\end{equation}
with material density $\rho_\text{s}$ (kg$\cdot$m$^{-3}$), cross-sectional area $A = \pi r^2$ (m$^2$), radius $r$ (m), tension $T = (2f_{0,\text{s}})^2\rho_\text{s}A$ (N), fundamental frequency $f_{0,\text{s}}$ (s$^{-1}$), Young's modulus $E_\text{s}$ (Pa), moment of inertia $I=\pi r^4 / 4$ (m$^4$), and loss coefficients $\sigma_{0,\text{s}}$ (s$^{-1}$) and $\sigma_{1,\text{s}}$ (m$^2$/s). We set the boundary conditions to be clamped so that
\begin{equation}\label{boundary}
    u_\text{s} = \partial_xu_\text{s} = 0, \quad \text{where} \quad x = \{0, L\}.
\end{equation}
As the string is excited using a bow, Equation \eqref{eq:PDEform} is rewritten to
\begin{equation}\label{eq:bowedSting}
    \mathcal{L}_\text{s}u_\text{s} = -\delta(x-x_\text{b})F_\text{b}\Phi(v_\text{rel}),
\end{equation}
with externally supplied bowing force $F_\text{b} = F_\text{b}(t)$ (N), spatial Dirac delta function $\delta(x-x_\text{b})$ selecting the bowing position $x_\text{b} = x_\text{b}(t)\in \mathcal{D}_\text{s}$ (m), dimensionless friction characteristic
\begin{equation}
    \Phi(v_\text{rel}) = \sqrt{2a}v_\text{rel}e^{-av_\text{rel}^2+1/2},
\end{equation}
with free parameter $a$ and the relative velocity between the string at bowing location $x_\text{b}$ and the externally supplied bowing velocity $v_\text{b} = v_\text{b}(t)$ (m/s):
\begin{equation}
    v_\text{rel} = \partial_tu_\text{s}(x_\text{b},t) - v_\text{b}.
\end{equation}

\subsection{Bridge}
The bridge is modelled as a simple mass with a linear restoring force. As this system is zero-dimensional, the state variable $u = u_\text{m}(t)$ and the definition of domain $\mathcal{D}$ is unnecessary. The operator $\mathcal{L}=\mathcal{L}_\text{m}$ is defined as
\begin{equation}
    \mathcal{L}_\text{m}=M\partial_t^2+M\omega_0^2+MR\partial_t,
\end{equation}
with mass $M$ (kg), linear angular frequency of oscillation $\omega_0=2\pi f_{0,\text{m}}$,  (s$^{-1}$), fundamental frequency $f_{0,\text{m}}$ (s$^{-1}$) and damping coefficient $R$ (s$^{-1}$).
\subsection{Body}
The body is simplified to a two-dimensional plate with side-lengths $L_x$ and $L_y$, domain $\mathcal{D} = \mathcal{D}_\text{p} = [0,L_x] \times [0,L_y]$ and state variable $u = u_\text{p}(x,y,t)$. Using the 2D Laplacian
\begin{equation}
    \Delta \triangleq \partial_x^2+\partial_y^2,
\end{equation}
the operator $\mathcal{L} = \mathcal{L}_\text{p}$ can be defined as
\begin{equation}
    \mathcal{L}_\text{p} = \rho_\text{p}H\partial_t^2 + D\Delta\Delta +2\rho_\text{p}H\sigma_{0,\text{p}}\partial_t-2\rho_\text{p}H\sigma_{1,\text{p}}\partial_t\Delta,
\end{equation}
with material density $\rho_\text{p}$ (kg$\cdot$m$^{-3}$), plate thickness $H$ (m), stiffness coefficient $D = E_\text{p}H^3/12(1-\nu^2)$, Young's modulus $E_\text{p}$ (Pa) and dimensionless Poisson's ratio $\nu$, and loss coefficients $\sigma_{0,\text{p}}$ (s$^{-1}$) and $\sigma_{1,\text{p}}$ (m$^2$/s). As with the stiff string, the boundary conditions of the plate are set to be clamped so that
\begin{equation}
    u_\text{p} = {\bf n} \cdot \nabla u_\text{p} = 0.
\end{equation}
where $\nabla u$ is the gradient of $u$.
\subsection{Collisions}
It can be argued that the greatest contriubutor to the characteristic sound of the tromba marina is the rattling bridge colliding with the body. A collision can be modelled by including a term to the PDEs described above describing the potential energy of the system (further referred to as \textit{the potential}) \cite{Ducceschi2019}
\begin{equation}\label{eq:potential}
    \phi(\eta) = \frac{K}{\alpha+1}[\eta]_+^{\alpha+1}, \quad K \geq 0,\quad \alpha \geq 1,
\end{equation}
where $K$ is the collision stiffness (N/m), $\alpha$ is the non-linear collision coefficient \textbf{check units here..}, and $\eta$ is the distance between the colliding objects (m). Furthermore, $[\eta]_+ = 0.5(\eta+|\eta|)$ is the positive part of $\eta$. The term which can then be included in the schemes is $\phi'(\eta) = d\phi/d\eta$. As described in \cite{Ducceschi2019}, using this form of the potential requires using iterative methods for solving its discrete counterpart. The authors proposed to rewrite the potential to
\begin{equation}
    \psi = \sqrt{2\phi},
\end{equation}
and the term included in the schemes to
\begin{equation}
    \phi' = \psi\psi' = \psi\frac{d\psi}{d\eta}
\end{equation}
which, as can be seen in Section \ref{sec:disc}, can be explicitly calculated. 

As the string rests on the bridge, the interaction between these components needs to be modelled as well. We can use an alternative version of the potential in Equation~\eqref{eq:potential} described in \cite{Bilbao2019} to make the collision two-sided acting as a connection:
\begin{equation}
    \phi(\eta) = \frac{K}{\alpha+1}|\eta|^{\alpha+1}.
\end{equation}
%
Using subscripts `$\text{sm}$' and `$\text{mp}$' to denote string-mass and mass-plate potentials, and including the effect of the bow from Equation \eqref{eq:bowedSting}, Equation \eqref{eq:PDEform} for all components can be rewritten to get the complete system:
\begin{subnumcases}{\label{eq:potentials}}
% \begin{aligned}
    \mathcal{L}_\text{s}u_\text{s} &$=-\delta(x-x_\text{b})F_\text{b}\Phi(v_\text{rel}) + \psi_\text{sm}\psi_\text{sm}'$,\qquad \label{eq:stringPotential}\\
    \mathcal{L}_\text{m}u_\text{m} &$= -\psi_\text{sm}\psi_\text{sm}' + \psi_\text{mp}\psi_\text{mp}',$\label{eq:massPotential}\\
    \mathcal{L}_\text{p}u_\text{p} &$= -\psi_\text{mp}\psi_\text{mp}',$\label{eq:platePotential}\\
    \eta_\text{sm} &$= u_\text{m} - u_\text{s}(x_\text{sm}, t),$\\
    \eta_\text{mp} &$=  u_\text{p}(x_\text{mp}, y_\text{mp}, t) - u_\text{m},$
% \end{aligned}
\end{subnumcases}
where $x_\text{sm} \in \mathcal{D}_\text{s}$ is the location of the bridge along the string and $(x_\text{mp}, y_\text{mp}) \in \mathcal{D}_\text{p}$ is where the bridge collides with the body.

Note that the potentials of belonging to a single collision 
in the above system have inverse signs as the collision force acts inversely on the two components. \textbf{not sure if this sentence is necessary, and otherwise rewrite..}
\section{Discretisation}\label{sec:disc}

\section{Implementation}
The real-time implementation of the system has been done in C++ using the JUCE framework \cite{JUCE2020}.
Initialisation is important. Both $\eta_\text{sm}$ and $\eta_\text{mp}$ need to be realistically chosen at sample $n=0$, i.e. $\eta_\text{sm}^0 = 0$ and $\eta_\text{mp}^0 \leq 0$.

Through empirical testing it was decided to retrieve the output from the state of the mass $u_\text{m}^n$ \textbf{(or the plate right at the point of collision $(x_\text{mp},y_\text{mp})$)} as the sound from the string and the plate were too dull. It can be argued that the loudest sound comes from the collision between the bridge and the body making it logical to select this point as the output position. 

\begin{tabular}{|l|c|c|}
    \hline
    Name & Symbol & Value\\ \hline
    \multicolumn{3}{|l|}{\bf String}\\ \hline
    Material density & $\rho_\text{s}$ & 7850\\ \hline
    Radius & $r$ & 0.005\\ \hline
    Fundamental frequency & $f_0$ & 50\\ \hline
    Young's modulus & $E_\text{s}$ & 2e11\\ \hline
    Frequency indep. loss & $\sigma_{0,\text{s}}$ & 0.1\\ \hline
    Frequency dep. loss & $\sigma_{1,\text{s}}$ & 1\\ \hline
    \multicolumn{3}{|l|}{\bf Bridge}\\\hline
    Mass & $M$ & 0.001\\ \hline
    Fundamental frequency & $f_{0,\text{m}}$ & 500\\ \hline
    Damping & $R$ & 0.1\\ \hline
    \multicolumn{3}{|l|}{\bf Body}\\ \hline
\end{tabular}

\section{Conclusions}
Parameter design
\begin{acknowledgments}
This work is supported by NordForsk's Nordic
University Hub Nordic Sound and Music Computing Network
NordicSMC, project number 86892.
\end{acknowledgments} 

%%%%%%%%%%%%%%%%%%%%%%%%%%%%%%%%%%%%%%%%%%%%%%%%%%%%%%%%%%%%%%%%%%%%%%%%%%%%%
%bibliography here
\bibliography{smc2020bib}

\end{document}
